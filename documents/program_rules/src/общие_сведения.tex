%\newpage
\vspace{1cm}
\section{Общие сведения}

Язык программирования C++\mdash компилируемый, статически типизированный язык общего назначения~\cite{Страуструп}, широко применяемый для написания исходных текстов программ.

Данный язык поддерживает такие парадигмы программирования, как процедурное программирование и объектно\sdash ориентированное программирование. Язык имеет богатую стандартную библиотеку, которая включает в себя распространённые контейнеры и алгоритмы, ввод\sdash вывод и многое другое.

Являясь одним из самых популярных языков программирования, С++ имеет широкую область применения от написания операционных систем и драйверов устройств до разнообразных прикладных программ. Процесс стандартизации языка C++ начался в 1989 году и продолжался до 1998 года, когда вышел стнадарт C++98 (ISO/IEC 17882-1998)\mdash за основу был взят язык в том виде, в котором он был описан его создателем \cite{Страуструп,СтандартнаяБиблиотекаДляПрофессионалов}.

При разработке в основном следует ориентироваться на компилятор GCC\sdash 8, почти полностью поддерживающий стандарт языка C++17. В большинстве задач достаточно использования стандартов C++11 и C++14.  

